\documentclass{article}
\usepackage{geometry}
\usepackage{fancyhdr}
\usepackage{lastpage}
\usepackage{amsmath}
\usepackage{textgreek}

\geometry{
    a4paper,
    total={170mm,257mm},
    left=20mm,
    top=20mm,
}

\pagestyle{fancy}
\fancyhf{}
\fancyhead[RE,LO]{Name:}
\fancyfoot[RE,LO]{Gesamtpunktzahl: 7P}
\fancyfoot[LE,RO]{Seite \thepage/\pageref{LastPage}}

\begin{document}

Fachbereich MND / WS 2021

\part*{Test 1 - Elektromagnetismus}

Die folgenden Aufgaben behandeln eine Luftspule mit der Windungszahl $N = 180$, dem Radius $r = 2\text{cm}$ und der Länge $l = 13\text{cm}$.

\subsection*{Berechnen Sie die Induktivität der Spule (4P)}

\begin{gather}
A = 2*\pi*r^{2} = 2,51*10^{-3}\text{m}^{2} \\
L = N^2*\mu_{0}*\frac{A}{l} = 786\text{\textmu H}
\end{gather}

\subsection*{Bestimmen Sie den magnetischen Widerstand $R_{m}$ der Spule (3P)}

\begin{gather}
R_{m} = \frac{l}{\mu_{0} * A} = 4,06*10^{7}\Omega
\end{gather}

\end{document}
