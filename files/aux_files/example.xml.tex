\documentclass{article}
\usepackage{wrapfig}
\usepackage{graphicx}
\usepackage{textgreek}
\usepackage{amsmath}
\usepackage{caption}
\usepackage{geometry}
\usepackage{fancyhdr}
\usepackage{lastpage}
\geometry{
    a4paper,
    total={170mm,257mm},
    left=20mm,
    top=20mm,
}

\pagestyle{fancy}
\fancyhf{}
\fancyhead[RE,LO]{Name:}
\fancyfoot[RE,LO]{Gesamtpunktzahl: 9P}
\fancyfoot[LE,RO]{Seite \thepage/\pageref{LastPage}}

\newcommand\floor[1]{\lfloor#1\rfloor}
\newcommand\ceil[1]{\lceil#1\rceil}

\graphicspath{ {./pictures/} }

\begin{document}

\part*{Beispielaufgabe - Thema Zusammenhangslosigkeit}

\begin{wrapfigure}{r}{0.3\textwidth}
\centering
\includegraphics[width=.95\linewidth]{ryz}
\caption*{Symbolbild}
\end{wrapfigure}
Hier gehört eine Übersicht der Aufgabe hin, falls vorhanden. Die folgenden Aufgaben sind Beispiele was mit diesem Programm möglich ist. Werden Werte in der Vorlage geändert, werden diese automatisch im Text (z.B.: $l= 3$m) sowie in den Rechnungen (z.B.: $A= 0.785398\text{mm}^2$) und Lösungen übernommen.\\
\subsection*{a) Berechnen Sie den Widerstand $R$ eines runden Kupferleiters mit der Leitfähigkeit $\kappa_{Cu}= 58\frac{\text{m}}{\Omega\text{mm}^2}$, der Länge $l = 3$m und dem Durchmesser $d = 1$mm. (4P)}

\vspace{15mm}\vspace{15mm}
\subsection*{b) Die Länge $l= 3$m wurde mit einem Lineal mit Millimeterskala gemessen. Wie lautet die absolute Messunsicherheit $\Delta x_a$ und die relative Messunsicherheit $\Delta x_r$? (2P)}

\vspace{15mm}\vspace{15mm}\vspace{15mm}
\subsection*{c) An den Kupferleiter wird eine ideale Spannungsquelle mit $U_q= 1$V angeschlossen. Bestimmen Sie den Strom, der durch den Leiter fließt. (3P)}

\vspace{15mm}


\end{document}
